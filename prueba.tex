\documentclass{article}
\usepackage[spanish,es-tabla, shorthands=off]{babel}
\usepackage[a4paper,top=2cm,bottom=2cm,left=3cm,right=3cm,marginparwidth=1.75cm]{geometry}
\usepackage{amsmath}
\usepackage{graphicx}
\usepackage{pgfplots}
\pgfplotsset{compat=1.17}
\usepackage{tikz}
\usepackage{array}
\usepackage{setspace}
	
\begin{document}
	
	\begin{titlepage}
		\centering
		\hfill
		\vspace{3cm}
		
		\hrule
		\vspace{1cm}
		\onehalfspacing
		{\Huge Impacto de la concurrencia en el rendimiento del algoritmo de ordenación \textit{Merge Sort}\par}
		\vspace{1cm}
		\hrule
		
		\vspace{1cm}
		{\LARGE ¿En qué medida las técnicas de concurrencia de Java optimizan el \textit{Merge Sort} iterativo y el recursivo? \par}
		
		\vspace{1cm}
		
		{\scshape\large Monografia de Informática\par}
		
		\vspace{8cm}
		
		{\large Cómputo de palabras: xxxx \par}
		{\large Código del alumno: lqv708 \par}
		
		
	\end{titlepage}
	
	
	
	
	
	
	
	
	
	
	
	
	\begin{figure}
		\centering
		\begin{tikzpicture}
			\begin{axis}[
				width=8cm,
				height=6cm,
				ymin=0, xmin=0,
				ymax=45, xmax=60,
				]
				\addplot[
					domain=0:8, 
					samples=100, 
					color=blue,
					line width=1pt
				]{x!};
			\end{axis}
		\end{tikzpicture}
		\label{fig:timeComplexities}
		\caption{Complejidades temporales}
	\end{figure}
	
	\begin{tikzpicture}
		\begin{axis}[
			ticks = none,
			width=8cm,
			height=6cm,
			restrict y to domain=0:10,
			restrict x to domain=0:10,
			domain = 0.00:10,
			xmin = 0,
			xmax = 11,
			ymin = 0,
			ymax = 11.5,
			]
			\addplot [
			samples=100, 
			color=red,
			line width=1pt,
			]
			{x^2}node[above right=-2pt,,pos=1]{$O(n^2)$};
			
			\addplot [
			samples=100, 
			color=blue,
			line width=1pt,
			]
			{x}node[above,pos=1]{${O}(n)$};
			\addplot [
			samples=100, 
			color=orange,
			line width=1pt,
			]
			{log2 x}node[above left,pos=1]{${O}(\log{}n)$};
			
			\addplot [
			samples=100, 
			color=black,
			line width=1pt,
			]
			{x*(log2 x)}node[right,pos=1]{${O}(n\log{}n)$};
			
			\addplot [
			samples=100, 
			color=magenta,
			line width=1pt,
			]
			{1}node[above,pos=1]{O$(1)$};
			
			\addplot [
			samples=100, 
			color=cyan,
			line width=1pt,
			]
			{x^3}node[above,pos=1]{${O}(n^3)$};
			
		\end{axis}
	\end{tikzpicture}
	
	
	
	
	
	
	
	
	
	\begin{tikzpicture}
		\begin{axis}[
			xmin=0, xmax=8,
			ymin=0, ymax=5,
			axis lines=middle,
			xlabel=n, ylabel=Tiempo,
			legend pos=north west,
			legend style={font=\small},
			grid=both,
			width=8cm,
			height=6cm,
			xticklabels={$n_0$},
			xtick=2,
			ytick=\empty
			]
			\addplot[
			domain=0:8, 
			samples=100, 
			color=blue,
			line width=1pt
			]
			{0.06816*x^5 - 1.25301*x^4 + 8.58526*x^3 - 26.82104*x^2 + 37.54837*x - 17.12774};
			\node[right] at (axis cs:6,3) {$f(x)$};
			
			\addplot[
			domain=0:8, 
			samples=100, 
			color=red,
			line width=1pt
			]
			{0.5*x^2 + 0.25*x - 1};
			\node[right] at (axis cs:5.5,1.5) {$\Omega(f(x))$};
			
			\addplot[
			domain=0:8, 
			samples=100, 
			color=green,
			line width=1pt
			]
			{-0.00556*x^2 + 0.14167*x + 0.43056};
			\node[right] at (axis cs:3.2,4.5) {$O(f(x))$};
			
			
			\addplot[ 
			color=black, dashed, thick 
			] coordinates{(2,0) (2,1.5)};
		\end{axis}
	\end{tikzpicture}


	
\end{document}


